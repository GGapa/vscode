% LAB2 %
\documentclass[a4,10pt]{article}

\usepackage{ctex}
\usepackage{graphicx}
\usepackage{subfigure}
\usepackage[top=20mm,bottom=20mm,left=20mm,right=20mm]{geometry}

\graphicspath{{figures/}}

\pagestyle{plain}

\begin{document}

% 标题区域开始
\begin{center}

\zihao{2}四川大学实验报告 \\
\vspace{1em}

\begin{tabular}{lll}
\zihao{4}学\hspace{1em}号\hspace{1em}N/A\hspace{1em} & \zihao{4}姓\hspace{1em}名\hspace{1em}辛明卿\hspace{1em}

\zihao{4}专\hspace{1em}业\hspace{1em}计算机\hspace{1em} & \zihao{4}班\hspace{1em}级\hspace{1em}XXX\hspace{1em}
\end{tabular}
    
\hrulefill

\begin{tabular}{lll}
\zihao{4}课程名称:\hspace{1em}Spatial Data Analysis\hspace{1em} & \zihao{4}实验课时:\hspace{1em}1\hspace{1em} \\

\zihao{4}实验项目:\hspace{1em}R Tutorial\hspace{1em} & \zihao{4}实验时间:\hspace{1em}2H\hspace{1em}
\end{tabular}

\end{center}
% 标题区域结束

\section{实验目的}
\large{} 
\begin{enumerate}
\item To learn to use R.
\item To learn how to analyse a data set.
\end{enumerate}

\section{实验环境}
\begin{enumerate}
\item Operating System : Windows 10 x64 22H2
\item Using Software : R 4.4.1 and RStudio 2024.04.2 Build 764
\end{enumerate}

\section{实验内容(算法、程序、步骤和方法)}
\begin{center}
\textbf{Assignment I} \\
Include the plots of kernel estimation with $\sigma = 0.05$ and $\sigma = 0.1$.
\end{center}
\begin{enumerate}
\item Add the data set: data(japanesepines).
\item Set another name: jp = japanesepines.
\item Use density.ppp() function to generate the intensity using
the isotropic Gaussian kernel : \newline
jp.Z.05 = density.ppp(jp, 0.05) \newline
jp.Z.1 = density.ppp(jp, 0.1)

\item Then, start plot : \newline
par(mar=c(0,0,1,1)) \newline
tl.05 = expression(paste("Kernel Estimation of JP: ", sigma, " = 0.05")) \newline
tl.1 = expression(paste("Kernel Estimation of JP: ", sigma, " = 0.1")) \newline

plot(jp.Z.1, main=tl.1) \newline
points(jp,pch="+",col="6") \newline
plot(jp.Z.05, main="Kernel Estimation of JP: sigma = 0.05") \newline
points(jp,pch="+",col="6") \newline

\begin{figure}[htbp]
\subfigure[sigma = 0.05]{
\includegraphics[scale=0.25]{assignment1_sigma=0.05.png} \label{1}
}
\quad
\subfigure[sigma = 0.05 with points]{
\includegraphics[scale=0.25]{assignment1_sigma=0.05_2.png} \label{2} 
}
\quad
\subfigure[sigma = 0.1]{
\includegraphics[scale=0.25]{assignment1_sigma=0.1_1.png}\label{3}
}
\quad
\subfigure[sigma = 0.1 with points]{
\includegraphics[scale=0.25]{assignment1_sigma=0.1_2.png}\label{4}
}
\caption{Experimental results of the authors}
\end{figure}

\end{enumerate}
\begin{center}
\textbf{Assignment II} \\
Create plots like Figure 1 and Figure 2 without specifying xlim argument (Plot function will use recommended range which is specified in the ppp object). Include plots in your report.
\end{center}
\begin{enumerate}
\item First, let’s plot G and F estimates diagrams : jp.ghat = Gest(jp) \newline
\item Define max value : g.max = max(jp.ghat\$r) \newline
\item Then, plot G estimates without xlim : plot(jp.ghat,cbind(rs,theo)\~r, main="G Estimates", xlab="r", ylab="G(r)")

\begin{figure}[htbp]
\includegraphics[scale=0.25]{assignment2_plot1.png}
\caption{Ghat Plot}
\label{2}
\end{figure} 

\item And then for F estimates :

\begin{figure}[htbp]
\includegraphics[scale=0.25]{assignment2_plot2.png}
\caption{Fhat Plot}
\label{3}
\end{figure} 

\item To plot K function diagram, we will need to use Kest() function : \newline
jp.khat = Kest(jp) \newline
plot(jp.khat, cbind(border, theo)~r, main="K function for JP data")

\begin{figure}[htbp]
\includegraphics[scale=0.25]{assignment2_plot3.png}
\caption{Khat Plot}
\label{4}
\end{figure} 

\item Similarly for L function diagram : 
\begin{figure}[htbp]
\includegraphics[scale=0.25]{assignment2_plot4.png}
\caption{Lhat Plot}
\label{5}
\end{figure} 

\item From the curve of the plot, easy to find out Ghat is clustered, Fhat and Khat are regular, Lhat is random. 
\end{enumerate}
\begin{center}
\textbf{Assignment III} \\
Create a plot for Ghat and Fhat with simulating bounds for Japanese pine sapling and provide it in your report.
\end{center}
\begin{enumerate}
\item First, simulate the homogeneous spatial Poisson (CSR) point processes : \newline
jp.r1 = runifpoint(jp\$n)
\item Then, 

\end{enumerate}

\section{数据记录与计算}
这里写你用到的数据

\section{结论(结果)}
描述实验结果

\section{小结}
这里写你的小结

\section{指导老师评议}
指导老师评议

\end{document} 